\documentclass[12pt]{article}
\usepackage{amsmath}
\begin{document}
Consider an array where the elements are with incremental natural numbers starting with 1 at the top right corner and following a diagonal pattern:

\begin{equation}
    a = 
\begin{bmatrix}
    1 & 2 & 4 & 7 & 11 & 16 & 22 & 29 \\
    3 & 5 & 8 & 12 & 17 & 23 & 30 & 38\\
    6 & 9 & 13 & 18 & 24 & 31 & 39 & 48\\
    10 & 14 & 19 & 25 & 32 & 40 & 49 & 59\\
    15 & 20 & 26 & 33 & 41 & 50 & 60 & 71\\
    21 & 27 & 34 & 42 & 51 & 61 & 72 & 84\\
    28 & 35 & 43 & 52 & 62 & 73 & 85 & 98
\end{bmatrix}\nonumber
\end{equation}

What follows is the procedure to obtain a formula that gives the value for $a_(ij)$ without having to find it iteratively.

Let $b$ be the first row of $a$:
\begin{equation}
    b =
\begin{bmatrix}
    1 & 2 & 4 & 7 & 11 & 16 &\cdots\nonumber\\
\end{bmatrix}
\end{equation}

Each number has a relationship with the previous one

\begin{eqnarray*}
b[0] &=& 1 + 0 = 1\\
b[1] &=& 1 + 1 = 2\\
b[2] &=& 2 + 2 = 4\\
b[3] &=& 4 + 3 = 7\\
b[4] &=& 7 + 4 = 11\\
b[5] &=& 11 + 5 = 16\\
\cdots&\\
b[n] &=& b[n-1] + n\\
\end{eqnarray*}

Extending $b[n-1]$ on each line shows a pattern:

\begin{eqnarray}
b[0] &=& 1\nonumber\\
b[1] &=& b[0] + 1 = 0(n-1) + 1 + n = 1n + 1\nonumber\\
b[2] &=& b[1] + 2 = 1(n-1) + 1 + n = 2n + 0\nonumber\\
b[3] &=& b[2] + 3 = 2(n-1) + 0 + n = 3n - 2\nonumber\\
b[4] &=& b[3] + 4 = 3(n-1) - 2 + n = 4n - 5\nonumber\\
b[5] &=& b[4] + 5 = 4(n-1) - 5 + n = 5n - 9\nonumber\\
b[6] &=& b[5] + 6 = 5(n-1) - 9 + n = 6n - 14\nonumber\\
\cdots&\nonumber\\
b[n] &=& n^2 - c[n] \label{eq:b}
\end{eqnarray}

where

\begin{equation}
c = \begin{bmatrix}-1 & -1 & 0 & 2 & 5 & 9 & 14 & \cdots\end{bmatrix}\nonumber
\end{equation}    
\begin{eqnarray*}
c[0] &=& -1\\
c[1] &=& -1 = 0 - 1\\
c[2] &=&  0 = 1 - 1\\
c[3] &=&  2 = 2 + 0\\
c[4] &=&  5 = 3 + 2\\
c[5] &=&  9 = 4 + 5\\
c[6] &=& 14 = 5 + 9\\
\cdots&\\
c[n] &=& (n-1) + c[n-1]\\
\end{eqnarray*}

Let's develop one, say $c[6]$, to unveil the pattern:

\begin{eqnarray*}
c[6] &=& (6-1) + c[5]\\
     &=& (6-1) + (5-1) + c[4]\\
     &=& (6-1) + (5-1) + (4-1) + c[3]\\
     &=& (6-1) + (5-1) + (4-1) + (3-1) + c[2]\\
     &=& (6-1) + (5-1) + (4-1) + (3-1) + (2-1) + c[1]\\
     &=& (6-1) + (5-1) + (4-1) + (3-1) + (2-1) + (1-1) + c[0]\\
\end{eqnarray*}
that looks like:

\begin{eqnarray} 
c[n] &=& \sum_{i=1}^{n} (i-1) + c[0]\nonumber
= \sum_{i=1}^{n}i - \sum_{i=1}^{n} 1 + c[0]\nonumber\\
     &=& \frac{n(n+1)}{2} - n -1\nonumber
     = \frac{n^2 + n - 2n - 2}{2}\nonumber\\
     &=& \frac{n^2 - n - 2}{2}\label{eq:c}\nonumber\\
\end{eqnarray}

And from (\ref{eq:b}) and (\ref{eq:c}) we have:

\begin{eqnarray}
b[n] &=& n^2 - \frac{n^2 - n - 2}{2}\nonumber
     = \frac{2n^2 - n^2 + n + 2}{2}\nonumber\\
     &=& \frac{n^2 +n +2 }{2}\label{eq:row}\nonumber\\
\end{eqnarray}

That gives us the formula to find out the elements of the first row.
To apply it to the rest of the rows, in (\ref{eq:row}) $n = i j$ and we offset the elements with the row $i$:

\begin{equation}
    a_{ij} =  i + \frac{(ij)^2+(ij)+2}{2}\label{eq:row1}\nonumber\\
\end{equation}
\end{document}